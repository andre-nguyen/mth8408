
\section{Description de la problématique choisie}
Pour ce projet nous avons choisi de reproduire la méthodologie et les résultats de l'article scientifique intitulé \textit{Minimum Snap Trajectory Generation and Control for Quadrotors} par Daniel Mellinger et Vijay Kumar. Dans cet article, les auteurs avaient entres autres l'objectif de générer des trajectoires qui prennent avantage de la dynamique des quadricoptères plutôt que de voir la dynamique en tant que contrainte sur le système \cite{Mellinger2011}. 

Lors du suivi d'une trajectoire, une solution triviale est souvent utilisée qui consiste en l'interpolation en ligne droite entre chaque point de cheminement aussi nommé \textit{waypoint}. Ceci est inneficace car la courbure infinie à chaque waypoint oblige le quadricoptère à s'arrêter avant de passer au prochain waypoint. Mellinger propose donc de modéliser une trajectoire optimale par un polynôme défini par parties entièrement lisse à travers les différents waypoints tout en satisfaisant des contraintes sur les vitesses et accélérations possibles du véhicule. Ce problème est résolu en le reécrivant en problème d'optimisation quadratique.

Tout dabord Mellinger démontre que la dynamique d'un quadricoptère a la propriété dêtre différentiellement plat. C'est-à-dire que les états et les entrées peuvent être exprimées par quatre sorties plates et leurs dérivées. Nous avons donc le vecteur de sorties plates
$$\sigma = [x, y, z, \psi]^T$$
où $r = [x, y, z]^T$ est la position du centre de masse dans le système de coordonnées du monde et $\psi$ l'angle de lacet. Une trajectoire est définie comme était une courbe lisse dans l'espace des sorties plates:
$$ \sigma(t) : [t_0, t_m] \rightarrow \mathbb{R}^3 \times SO(2)$$ 
où $t_0$ et $t_m$ sont les temps de début et de fin de la trajectoire, $m$ correspond au nombre d'intervales de temps entre chaque waypoint et $SO(2)$ est le groupe spécial orthogonal. En pratique, une trajectoire est plutôt décrite par un polynôme défini par parties:
\begin{align}\label{eq:polynomial}
\sigma_T(t) =
\left\{
	\begin{array}{ll}
		\sum_{i=0}^n \sigma_{Ti1} t^i  & t_0 \leq t < t_1 \\
		\sum_{i=0}^n \sigma_{Ti2} t^i  & t_1 \leq t < t_2 \\
		... \\
		\sum_{i=0}^n \sigma_{Tim} t^i  & t_{m-1} \leq t < t_m \\
	\end{array}
\right.
\end{align}
où $n$ est l'ordre du polynôme et $m$ est encore le nombre d'intervales de temps.

\subsection{Formulation du problème de génération de trajectoire}

Au final, le but premier de la méthode de Mellinger est d'optimiser la dérivée d'ordre $k_r$ de la position au carré et la dérivée d'ordre $k_\psi$ de l'angle de lacet du véhicule au carré.
\begin{align}\label{eq:opt}
\text{min} \int_{t_0}^{t_m} \mu_r \norm{\frac{d^{k_r} r_T}{dt^{k_r}}}^2 + \mu_\psi \frac{d^{k_\psi} \psi_T}{dt^{k_\psi}}^2
\end{align}\begin{align*}
	\begin{array}{lll}
		\text{sous contraintes} & \sigma_T(t_i) = \sigma_i & i = 0, \ldots, m\\
		& \frac{d^p x_T}{dt^p}|_{t=t_j} = 0\ \text{ou libre,} & j = 0, \ldots, m; p = 1, \ldots, k_r\\
		& \frac{d^p y_T}{dt^p}|_{t=t_j} = 0\ \text{ou libre,} & j = 0, \ldots, m; p = 1, \ldots, k_r\\
		& \frac{d^p z_T}{dt^p}|_{t=t_j} = 0\ \text{ou libre,} & j = 0, \ldots, m; p = 1, \ldots, k_r\\
		& \frac{d^p \psi_T}{dt^p}|_{t=t_j} = 0\ \text{ou libre,} & j = 0, \ldots, m; p = 1, \ldots, k_\psi\\
	\end{array}
\end{align*}

où $\mu_r$ et $\mu_\psi$ sont des constantes qui rendent l'intégrale non dimensionelle, $\sigma_T = [x_T, y_T, z_T, \psi_T]^T$ et $\sigma_i = [x_i, y_i, z_i, \psi_i]^T$. Mellinger et Kumar choisissent $k_r = 4$ c'est-à-dire la deuxième dérivée de l'accélération (le \textit{snap}) et $k_\psi = 2$. En d'autres mots, les contraintes expriment les waypoints à travers lesquels le véhicule doit voler et la vitesse, l'accélération, le \textit{jerk} et le \textit{snap} désiré à chaque waypoint.

Le problème peut être formulé en tant que problème d'optimisation quadratique en réécrivant les constantes $\sigma_{T_{ij}} = [x_{T_{ij}}, y_{T_{ij}}, z_{T_{ij}}, \psi_{T_{ij}}]$ en un vecteur $c$ de dimension $4nm \times 1$ avec les variables de décision $\{x_{T_{ij}}, y_{T_{ij}}, z_{T_{ij}}, \psi_{T_{ij}}\}$ pour avoir la forme standard:
\begin{align}\label{eq:opt_quad}
\text{min}\ \ \ c^THc+f^Tc
\end{align}\begin{align*}
	\begin{array}{ll}
	\text{s. c.} & Ac\leq b
	\end{array}
\end{align*}

Le vecteur $c$ est donc le vecteur contenant les coefficients des polynômes définissant la trajectoire. 

Mellinger démontre aussi une façon de modifier le problème d'optimisation pour ajouter des contraintes de corridor. Autrement dit, c'est une façon de forcer la trajectoire à respecter un corridor de sécurité entre deux waypoints pour éviter une collision.

\subsection{Formulation du problème d'allocation de temps}

Si jamais le temps d'arrivé à chaque waypoint intermédiaire importe peu, il est possible de réallouer les temps assignées entre chaque waypoint. Une fois que la solution $f(T)$ de (\ref{eq:opt}) est trouvée avec un pas de temps égal entre chaque waypoint, on résous un autre problème d'optimisation
\begin{align}\label{eq:time_opt}
\text{min}\ \ \ f(T)
\end{align}\begin{align*}
\begin{array}{lll}
\text{s. c.} & \sum T_i = t_m & i = 1,\ldots,m\\
& T_i \geq 0 &  i = 1,\ldots,m\\
\end{array}
\end{align*}
où $T_i = t_i - t_{i-1}$ sont les temps alloués à chaque segment de la trajectoire. L'optimisation se fait au moyen d'une descente du gradient avec un "backtracking line search".

\section{Objectifs et plan d'action}

Le but général du projet est de reproduire la méthode complète de génération de trajectoire et si le temps le permet, de brancher l'implémentation dans Robot Operating System et le simulateur de physique Gazebo et faire voler une trajectoire par un multirotor en simulation. Plus précisément, les objectifs en ordre de priorité sont:
\begin{enumerate}
	\item Modéliser le problème et résoudre (\ref{eq:opt_quad}) pour générer une trajectoire lisse préléminaire en Matlab au moyen du solveur \textit{quadprog} en suivant les équations dans l'article
		\begin{itemize}
			\item La difficulté ici est de formuler une méthode pour automatiquement construire la matrice hessienne $H$ pour $m \  \epsilon \ \mathbb{Z}$ et les matrices représentant les contraintes.
		\end{itemize}
	\item Résoudre le problème de réallocation des temps (\ref{eq:time_opt}) avec une implémentation "fait maison" de la descente du gradient en Matlab
	\item Ajouter les contraintes de corridor de sécurité à (\ref{eq:opt_quad}) en Matlab
	\item Si le temps le permet: Transférer le code à C++ ou Python pour brancher le générateur de trajectoire à une simulation réaliste
		\begin{itemize}
			\item Dans le cas de C++ il devrait être possible d'utiliser la librairie d'algèbre linéaire Eigen et le solveur OOQP.
		\end{itemize}
\end{enumerate}

En ce qui concerne la dernière étape du projet, le simulateur de drones développé par Furrer et al. au moyen de Robot Operating System et du simulateur Gazebo sera utilisé \cite{Furrer2016}. Cette simulation inclut entres autres plusieurs modèles de multirotors et une implémentation d'un contrôleur de trajectoires basé sur le contrôleur gémétrique proposé par Lee et al. \cite{Lee2010} qui sera utilisé au lieu du contrôleur décrit dans l'article et Mellinger.

\section{Impact attendu}

Le laboratoire de robotique mobile et des systèmes autonomes est souvent appelé à faire des présentations grand public à fin de représenter Polytechnique Montréal lors des évènements portes ouvertes ou lors des visites de partenaires industriels. Dans l'idéal, la complétion de ce projet permettrait de générer de belles trajectoires (lisses) pouvant être volées par l'un des véhicules multirotors du laboratoire lors de ces présentations.

Bien qu'il existe maintenant des méthodes plus performantes et plus complexes pour la génération de trajectoires par optimisation tel que \cite{Richter2016} et \cite{Chen2016}, elles reposent toujours sur une forme d'optimisation quadratique similaire à celle de Mellinger dans \cite{Mellinger2011}. L'implémentation de cette méthode est donc un moyen idéal d'étudier la formulation la plus simple d'un problème d'optimisation quadratique dans le contexte de la génération de trajectoires pour multirotors dans le but d'aussi comprendre les méthodes modernes.