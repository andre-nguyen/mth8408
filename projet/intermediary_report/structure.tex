\section{Structure du problème d'optimisation}

Considérons le problème en une dimension $p$, où $p = x$, $y$ ou $z$. Nous pouvons reformuler le problème en tant que problème d'optimisation quadratique en réécrivant les coefficients des polynômes en un vecteur $c$ de dimension $4m \times 1$, où $m$ est le nombre de waypoints en excluant les conditions initiales. Nous obtenons la forme standard:
\begin{align}\label{eq:opt_quad}
\text{min}\ \ \ c^THc+f^Tc
\end{align}\begin{align*}
	\begin{array}{ll}
	\text{s. c.} & Ac = b
	\end{array}
\end{align*}

Avant de construire la matrice $H$ nous devons développer les équations des polynômes représentant un segment de trajectoire i.e. la trajectoire entre deux waypoints. En réglant l'ordre du polynôme à $n = 6$ nous avons pour un axe de déplacement $p$ en 3D où $p = x$, $y$ ou $z$:
\begin{align*}
\boldsymbol{p} &= c_6 t^6 + c_5 t^5 + c_4 t^4 + c_3 t^3 + c_2 t^2+ c_1 t + c_0\\
\frac{d \boldsymbol{p}}{dt} &= 6 c_6 t^5 + 5 c_5 t^4 + 4 c_4 t^3 + 3 c_3 t^2 + 2c_2 t + c_1 + 0\\
\frac{d^2 \boldsymbol{p}}{dt^2} &=
	(6\cdot 5) c_6 t^4 + (5 \cdot 4) c_5 t^3 + (4 \cdot 3) c_4 t^2 + (3 \cdot 2) c_3 t + 2c_2 + 0 + 0\\
\frac{d^3 \boldsymbol{p}}{dt^3} &=
	(6\cdot 5\cdot 4) c_6 t^3 + (5 \cdot 4\cdot 3) c_5 t^2 + (4 \cdot 3\cdot 2 ) c_4 t + (3 \cdot 2) c_3 + 0 + 0 + 0\\
\frac{d^4 \boldsymbol{p}}{dt^4} &=
	(6\cdot 5\cdot 4\cdot 3) c_6 t^2 + (5 \cdot 4\cdot 3\cdot 2) c_5 t + (4 \cdot 3\cdot 2 ) c_4 + 0 + 0 + 0 + 0\\
\end{align*}
Prendre la norme euclidienne au carré de la position est équivalent à prendre le carré de chaque polynôme. Donc pour un polynôme $p$ représentant un axe $x$, $y$ ou $z$ nous avons:
\begin{align}\label{eq:polynome_derive}
\norm{\frac{d^4 \boldsymbol{p}}{dt^4}}^2 &=
	\bigg((6\cdot 5\cdot 4\cdot 3) c_6 t^2 + (5 \cdot 4\cdot 3\cdot 2) c_5 t + (4 \cdot 3\cdot 2 ) c_4 \bigg)^2 \\
	&=	(6\cdot 5\cdot 4\cdot 3)^2 c_6^2 t^4 + (6\cdot 5\cdot 4\cdot 3)(5 \cdot 4\cdot 3\cdot 2) c_6 c_5 t^3 + (6 \cdot 5 \cdot 4\cdot 3)(4\cdot 3\cdot 2)c_6 c_4 t^2 \nonumber\\
	&	+ (5 \cdot 4\cdot 3\cdot 2)^2 c_5^2 t^2 + (6\cdot 5\cdot 4\cdot 3)(5 \cdot 4\cdot 3\cdot 2) c_6 c_5 t^3 + (5 \cdot 4\cdot 3\cdot 2)(4\cdot 3\cdot 2)c_5 c_4 t \nonumber\\
	&	+ (4\cdot 3\cdot 2)^2 c_4^2 + (6 \cdot 5 \cdot 4\cdot 3)(4\cdot 3\cdot 2)c_6 c_4 t^2 + (5 \cdot 4\cdot 3\cdot 2)(4\cdot 3\cdot 2)c_5 c_4 t \nonumber\\
	& = (6\cdot 5\cdot 4\cdot 3)^2 c_6^2 t^4 + 2(6\cdot 5\cdot 4\cdot 3)(5 \cdot 4\cdot 3\cdot 2) c_6 c_5 t^3 + 2(6 \cdot 5 \cdot 4\cdot 3)(4\cdot 3\cdot 2)c_6 c_4 t^2 \nonumber\\
	&	+ (5 \cdot 4\cdot 3\cdot 2)^2 c_5^2 t^2 + 2 (5 \cdot 4\cdot 3\cdot 2)(4\cdot 3\cdot 2)c_5 c_4 t \nonumber\\
	&	+ (4\cdot 3\cdot 2)^2 c_4^2 \nonumber
\end{align}

Ensuite, pour un waypoint $j$ nous avons le temps de départ $t_j$ et le temps d'arrivée au prochain waypoint $t_{j+1}$, nous pouvons écrire le résultat de l'intégrale:
\begin{align}\label{eq:polynome_integre}
\int_{t_j}^{t_{j+1}} \norm{\frac{d^4 \boldsymbol{p}}{dt^4}}^2 dt
	& = \int_{t_j}^{t_{j+1}} (6\cdot 5\cdot 4\cdot 3)^2 c_6^2 t^4 + 2(6\cdot 5\cdot 4\cdot 3)(5 \cdot 4\cdot 3\cdot 2) c_6 c_5 t^3 \\
	&	+ 2(6 \cdot 5 \cdot 4\cdot 3)(4\cdot 3\cdot 2)c_6 c_4 t^2 + (5 \cdot 4\cdot 3\cdot 2)^2 c_5^2 t^2 + 2(5 \cdot 4\cdot 3\cdot 2)(4\cdot 3\cdot 2)c_5 c_4 t \nonumber \\
	&	+ (4\cdot 3\cdot 2)^2 c_4^2 \ dt\nonumber \\
%	&= (6\cdot 5\cdot 4\cdot 3)^2 c_6^2 \frac{1}{4} t^3 \Big|_{t_j}^{t_{j+1}} + 2(6\cdot 5\cdot 4\cdot 3)(5 \cdot 4\cdot 3\cdot 2) c_6 c_5 \frac{1}{3} t^2\Big|_{t_j}^{t_{j+1}}\nonumber \\
%	&	+ 2(6\cdot 5\cdot 4\cdot 3)(4\cdot 3\cdot 2) c_6 c_4 \frac{1}{2} t\Big|_{t_j}^{t_{j+1}}
%		+ (5 \cdot 4\cdot 3\cdot 2)^2 c_5^2 \frac{1}{2} t\Big|_{t_j}^{t_{j+1}} \nonumber \\
	&=	360^2 c_6^2 \frac{1}{5} t^5 \Big|_{t_j}^{t_{j+1}}
		+ 2 \cdot 360 \cdot 120 c_6 c_5 \frac{1}{4} t^4\Big|_{t_j}^{t_{j+1}}
		+ 2 \cdot 360 \cdot 24 c_6 c_4 \frac{1}{3} t^3\Big|_{t_j}^{t_{j+1}} \nonumber \\
	&	+ 120^2 c_5^2 \frac{1}{3} t^3\Big|_{t_j}^{t_{j+1}} + 2 \cdot 120 \cdot 24 c_5 c_4 \frac{1}{2}t^2 \Big|_{t_j}{t_{j+1}} \nonumber \\
	&	+ 24^2 c_4^2 t\Big|_{t_j}^{t_{j+1}} \nonumber
\end{align}


Avec le résultat en (\ref{eq:polynome_integre}) on peut maintenant poser une partie de la matrice $H$ de (\ref{eq:opt_quad}).

Tel qu'indiqué précédemment, le vecteur $c$ contient tous les coefficients de tous les polynômes de chaque segment de la trajectoire et de chaque degré de liberté. La partie de $c$ correspondant à un axe $p$ est donc $[c_6,\ c_5,\ c_4,\ c_3,\ c_2,\ c_1,\ c_0]^T$. Par conséquent, nous pouvons déduire la matrice $H_{pj}$ correspondante pour un waypoint $j$
\begin{align}\label{eq:hessienne_p}
H_{pj} =
\begin{bmatrix}
    360^2 \frac{1}{5} t^5 \Big|_{t_j}^{t_{j+1}}
    		& 2 \cdot 360 \cdot 120 \frac{1}{4} t^4\Big|_{t_j}^{t_{j+1}}
    		& 2 \cdot 360 \cdot 24 \frac{1}{3} t^3\Big|_{t_j}^{t_{j+1}}
    		& 0
    		& 0
    		& 0
	    	& 0 \\
    2 \cdot 360 \cdot 120 \frac{1}{4} t^4\Big|_{t_j}^{t_{j+1}}
    		& 120^2 \frac{1}{3} t^3\Big|_{t_j}^{t_{j+1}}
    		& 2 \cdot 120 \cdot 24 \frac{1}{2}t^2 \Big|_{t_j}^{t_{j+1}} & 0 & 0 & 0 & 0\\
	2 \cdot 360 \cdot 24 \frac{1}{3} t^3\Big|_{t_j}^{t_{j+1}}
		& 2 \cdot 120 \cdot 24 \frac{1}{2}t^2 \Big|_{t_j}^{t_{j+1}}
		& 24^2 t\Big|_{t_j}^{t_{j+1}} & 0 & 0 & 0 & 0 \\
    0 & 0 & 0 & 0 & 0 & 0 & 0 \\
    0 & 0 & 0 & 0 & 0 & 0 & 0 \\
    0 & 0 & 0 & 0 & 0 & 0 & 0 \\
    0 & 0 & 0 & 0 & 0 & 0 & 0 \\
\end{bmatrix}
\end{align}

Bien que Mellinger ne le présente pas dans son article, Richter et Bry \cite{Richter2016, bry2012control} indiquent que ce processus peut être automatisé par la formule suivante:
\begin{align}
H_{pj} =\left\{
  \begin{array}{ll}
    2 \Big( \prod_{m = 0}^{r-1} (i-m)(l-m)\Big) \frac{\tau^{i+l-2r+1}}{i+l-2r+1}: & i \geq r \text{ ou } l \geq r \\
    0 : & \text{autrement}
  \end{array}
  \right.
\end{align}
où $i$ et $l$ sont les index de rangée et de colonne et $\tau$ est la durée du segment de trajectoire.

\subsection{Structure de la matrice H finale}

Une fois le développement précédent fait, nous pouvons concaténer en diagonale les matrices $H_{pj}$ pour chaque waypoint jusqu'à $j=m$, pour chaque waypoint jusqu'au dernier waypoint $j=m$ pour former $H$ de (\ref{eq:opt_quad}).

\begin{align}
H_x=
\begin{bmatrix}
	H_{x1} \\
	&	H_{x2} \\
	&	&		\ddots \\
	&	&		&		H_{xm}
\end{bmatrix}
\end{align}

De plus, par le développement précédent en (\ref{eq:polynome_integre}), nous pouvons voir que le terme linéaire $f^Tc$ de (\ref{eq:opt_quad}) est nul.

\subsection{Structure des matrices de contraintes A et b}

Dans ce problème nous utilisons les matrices de contraintes d'égalité de deux façons. Premièrement pour imposer la valeur d'une certaine dérivée à un certain moment et deuxièmement pour imposer la continuité entre deux segments de trajectoire.

Dans notre code, nous avons appelé ceci les contraintes de départ et d'arrivée et les contraintes de continuité. Encore une fois, nous suivons la formulation de Bry \cite{bry2012control}(car Mellinger donne peu de détails dans son article) où pour un segment de trajectoire allant du temps $0$ au temps $tau$ nous pouvons contruires $A$ et $b$ de telle manière que
\begin{align}
A = \begin{bmatrix} A_0 \\ A_\tau \end{bmatrix}, b = \begin{bmatrix} b_0 \\ b_\tau \end{bmatrix}
\end{align}
