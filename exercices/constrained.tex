\section{Constrained Optimization}

\subsection{Problem 1}
\textbf{a)}

\begin{equation*}
\begin{aligned}
& \underset{x}{\text{minimize}}
& & 2 \pi r^2 + 2 h \pi r \\
& \text{subject to}
& & V = \pi h r^2
\end{aligned}
\end{equation*}

Write Lagrangian

\begin{align*}
	\mathcal{L}(x;\lambda) = 2 \pi r^2 + 2 h \pi r - \lambda(\pi h r^2-V)
\end{align*}

Gradient of Lagrangian is equal to 0, (first order optimality condition)

\begin{align*}
	\nabla \mathcal{L}(x;\lambda) = 
	\begin{bmatrix}
		2\pi r \\
		4 \pi r + 2 h \pi	
	\end{bmatrix} - \lambda
	\begin{bmatrix}
		\pi r^2 \\
		2 \pi h r
	\end{bmatrix} = \begin{bmatrix} 0 \\ 0	\end{bmatrix}
\end{align*}

We can isolate $\lambda$ using the first row of the gradient.

\begin{align*}
	2 \pi r - \lambda \pi r^2 &= 0 \\
	2 - \lambda r &= 0 \\
	\lambda r &= 2 \\
	\lambda = \frac{2}{r}
\end{align*}

Plugging into the second row we get

\begin{align*}
	4 \pi r + 2 h \pi - \lambda 2 \pi h r &= 0 \\
	2 r + h - \lambda h r &= 0 \\
	2 r + h - 2 h &= 0 \\
	h &= 2r
\end{align*}

Now using the equality constraint we get

\begin{align*}
	\pi h r^2 -V &= 0 \\
	\pi 2 r r^2 - V &= 0 \\
	V &= 2 \pi r^3 \\
	r = (\frac{V}{2\pi})^{\frac{1}{3}}
\end{align*}

Consequently we also get

\begin{align*}
	h &= 2 (\frac{V}{2\pi})^{\frac{1}{3}} \\
	\lambda &= \frac{2}{(\frac{V}{2\pi})^{\frac{1}{3}}}
\end{align*}

\textbf{b)}

Maximizing $V$ is equivalent to minimizing $-V$ thus we have the problem

\begin{equation*}
\begin{aligned}
& \underset{x}{\text{minimize}}
& & -V = - \pi r^2 h \\
& \text{subject to}
& & q = 2 \pi r^2 + 2 \pi r h
\end{aligned}
\end{equation*}

Write the Lagrangian function
\begin{align*}
	\mathcal{L}(x; \lambda) = -\pi r^2 h - \lambda(2\pi r^2 + 2 \pi r h)
\end{align*}

Se the gradient of the Lagrangian to $0$
\begin{align*}
	\nabla_x \mathcal{L}(x;\lambda) = \begin{bmatrix}
	-2 \pi r h \\ -\pi r^2
	\end{bmatrix} - \lambda \begin{bmatrix}
		4 \pi r + 2 \pi h \\ 2 \pi r
	\end{bmatrix} = \begin{bmatrix}
	 0 \\ 0
	\end{bmatrix}
\end{align*}

Using the first row we can isolate $\lambda$ 
\begin{align*}
	-2 \pi r h - \lambda (4 \pi r + 2 \pi h) &= 0\\
	-r h - \lambda r - \lambda h &= 0 \\ 
	-\lambda (r+H) &= rh \\
	\lambda = \frac{-rh}{r+h}
\end{align*}

We can do the samething usign the second row to write $\lambda$ as a function of $r$.
\begin{align*}
	- \pi r^2 - \lambda 2 \pi r &= 0 \\
	r - 2 \lambda &=0 \\
	r &= 2\lambda \\
	\lambda &= \frac{r}{2}
\end{align*}

Plugging this result into the previous equation we can write $r$ as a function of $h$
\begin{align*}
	\frac{r}{2} = - \frac{rh}{r+h} \\
	\frac{r+h}{2} = -h \\
	r = -2h -h \\
	r = -3h
\end{align*}

Plugging this result into the constraint equation we get

\begin{align*}
	2 \pi r^2 + 2 \pi r h &= q \\
	2 \pi (-3h)^2 + 2 \pi (-3 h) h  &= \\
	2 \pi (9 h^2 - 3h^2) &= \\
	2 \pi (6 h^2) &= \\
	h^2 &= \frac{q}{12 \pi} \\
	h &= \pm (\frac{q}{12 \pi})^{\frac{1}{2}}
\intertext{Using this we can get expressions for $r$ and $\lambda$}
	r &= \mp 3 \Big(\frac{q}{12 \pi}\Big)^{\frac{1}{2}} \\
	\lambda &= \mp \frac{3}{2} \Big(\frac{q}{12 \pi}\Big)^{\frac{1}{2}} \\
\end{align*}

\subsection{Problem 2}
\textbf{a)}

Maybe its a good idea because it relaxes the constraint a little bit and is easier to work with numerically?

\textbf{b)}

No because its more complex?

\subsection{Problem 3}

\begin{equation*}
\begin{aligned}
& \underset{x}{\text{minimize}}
& & -V = - \pi r^2 h \\
& \text{subject to}
& & q = 2 \pi r^2 + 2 \pi r h
\end{aligned}
\end{equation*}

